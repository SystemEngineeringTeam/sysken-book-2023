\section{戦闘機に乗ろう}
\subsection{DCS World}
\subsubsection{はじめに}
はじめまして、松土です。いきなりですが、僕はミリオタ\footnote{ミリタリーオタクの略 所謂軍事が好きな人}で、特に戦闘機が好きです。僕は戦闘機に乗りたい!!!。しかし、戦闘機というのは皆さんもご存知の通り、軍用の航空機のため主に軍隊が保有しており、日本では航空自衛隊のみが保有しています。
そんな戦闘機に乗ろうと思ったら、航空自衛隊に入隊し、高い倍率の選抜を受けた後に何年もの厳しい訓練を乗り越えないといけません。
そこで、一度は耳にしたことであろう、フライトシミュレーターというのがこの世には存在しており、これは一般人が仮想のコクピットに乗り込み、操縦をシミュレートできる素晴らしいものです。
本物のパイロットが実機を操縦する前の訓練をするにあたって使用する事もあります。僕はこれを利用して戦闘機に乗りたい欲を解消することにしました。
\subsubsection{DCS Worldとは}
いきなり出てきたDCS World\footnote{Degital Combat Simulator Worldの略}とは何か?、これは戦闘機に特化したフライトシミュレーターで、フランス空軍にも採用されている大変優れたものです。
現実に存在する戦闘機がいくつかモジュールとしてあり、一般人が家庭でパソコンの中にインストールするだけで、実機を仮想状で操縦できるようになります。
プラットフォームはMicrosoft Windowsで、2008年からリリースされており、開発元は本社をロシアのモスクワに置いているEagle Dynamics社です。


>>>アプリは無料です<<< ※モジュールは有料です
\subsection{動作環境}
\subsubsection{最小構成}
\begin{itemize}
  \item OS: 64-bit Windows 10 , DirectX11(version 2.7以降はWindows 7非対応かつWindows 8が明記落ち、version 2.5.xまではWindows 7/8も可)
  \item CPU: Intel Core i3 2.8 GHz / AMD FX
  \item RAM: 8 GB (重いミッションをプレイする場合: 16 GB)
  \item HDD空き容量: 60 GB (※注:2022年夏時点のversion2.7.16では120GBに増加している)
  \item ビデオカード: NVIDIA GeForce GTX 760 / AMD R9 280X以上
  \item ユーザー認証用インターネット接続
\end{itemize}

\subsubsection{推奨構成}
\begin{itemize}
  \item OS: 64-bit Windows 10 , DirectX11(version 2.7以降はWindows 8が明記落ち、version 2.5.xまではWindows 8も推奨内)
  \item CPU: Core i5 3GHz 以上 / AMD FX 又は Ryzen
  \item RAM: 16 GB (重いミッションをプレイする場合: 32 GB)
  \item HDD空き容量: 130 GB (SSD推奨 全モジュール導入には460GB)
  \item ビデオカード: NVIDIA GeForce GTX 1070 / AMD Radeon RX VEGA 56 (8GB VRAM) 以上
  \item ジョイスティック
  \item ユーザー認証用インターネット接続
\end{itemize}

\subsubsection{VRを使用する場合}
推奨構成に上書きして
\begin{itemize}
  \item CPU: Core i5 3GHz 以上 / AMD FX 又は Ryzen
  \item RAM: 16 GB (重いミッションをプレイする場合: 32 GB)
  \item HDD空き容量: 130 GB (SSD推奨 全モジュール導入には460GB)
  \item ビデオカード: NVIDIA GeForce GTX 1080 / AMD Radeon RX VEGA 64 (8GB VRAM) 以上
\end{itemize}

\subsection{導入}
\subsubsection{公式版とSteam版}
公式DCS Wolrdのサイトよりダウンロードする場合とゲームプラットフォームで有名なSteamでダウンロードする場合があります。
特に両者違いはありませんが、Steam版は公式版よりも更新が遅いことがあります。

\subsubsection{安定版とOpen Beta版}
新しいモジュールはまずOpen Beta版のみに対して提供され、何度かのOpen Betaアップデートを経て十分にバグを無くしてから安定版への提供となります。
  \textbf{Open Betaの利点}
マルチプレイサーバーの多くはOpen Betaを使用している(2023/05/07現在)ためマルチプレイをしたいのならOpen Beta推奨
新しいモジュールをより早く遊ぶことができる
レーダーやFLIRなど、前バージョンではモジュールへの実装がオミットされていた一部機能や兵装が追加実装されたのを早く体験できる

  \textbf{Open Betaの欠点}
Open Betaは当然バグが多く、特定の武器を使うとゲームがクラッシュする現象が起きることもある
ゲームがクラッシュまではしなくても、以前のバージョンでは正常だったはずの特定のミサイルの誘導能力がおかしくなっている、レーダーや照準などの装置の操作や挙動がおかしくなっている、といったバグが新しく増えていることもある。

\subsubsection{モジュールの購入}
アプリ自体は無料のため、インストール後起動し、プレイすることが可能ですが、
初期で乗ることのできる機体は、第二次世界大戦で使用されたプロペラ機の練習機版TF-51とソ連\footnote{ソビエト連邦の略}が開発した攻撃機Su-25だけです。
どちらも、戦闘機ではない上にカッコ悪い\footnote{あくまで個人の感想です}です。


注意として、機体内部のボタンやスイッチを一つ一つまで操作できる機体(クリッカブル機)と、キーボードを使い、キーで細かい機体の操作を行う機体(FC3機)があり、
クリッカブル機は基本的に高価ですが、実機と同じく全てのボタンが操作でき、リアルな操作を楽しむことができます。

\subsection{まとめ}
今回は、ミリオタ向けフライトシミュレーターとして、DCS Worldを紹介させていただきました。導入した後の楽しみ方は、実機と同じエンジンスタートを行ってみたり、マルチプレイで友人と飛んでみたり、または戦ってみたり、と様々であり、
あなたの思うがままにやりたいことができるのがこのDCS Worldの良いところです。もしも、これを読んでいるあなたがDCS Worldを初めたら、僕と一緒に飛びましょう。