\chapter{リポジトリ作成後に設定しておきたいこと}
\section{はじめに}
こんにちは。hihumikanです。

本チャプターでは「リポジトリ作成後に設定しておきたいこと」をご紹介致します。
私自身が次回ハッカソンに参加する際に、こういった設定をするだろうというものをまとめました。

ただし、これら全ての内容をプロジェクトに適用出来るものではないため、ご利用の環境や用途に合わせて利用していただければと思います。

\section{ファイルの設定}
リポジトリ作成後に設定しておきたい「ファイルの設定」についてご紹介します。

\subsection{.gitignore}
.gitignoreファイルは、Gitによる管理から除外したいファイルやディレクトリを指定するための設定ファイルです。

例えば、MacOSの場合、ディレクトリ毎に.DS\_Storeというファイルが生成されます。
このファイルは、ディレクトリのmeta情報を記録しますが、Gitの管理対象外としておくことが望ましいです。
\footnote{通常の開発において、共有する必要がないためである。}

下記のようなテキストファイルをGit配下に置くだけで、Gitから管理対象外として扱うことができます。
\begin{tcolorbox}[title=.gitignore]
  \begin{verbatim}
1 .DS_Store
\end{verbatim}
\end{tcolorbox}

必要に応じて.gitignoreファイルを編集し、管理対象外とするファイルやディレクトリを指定しましょう。

\subsection{.gitattributes}

\subsection{Makefile}
\subsection{pre-commit(git hooks)}


\section{インフラ回り}

\subsection{CI/CD}
\subsection{docker}
\subsection{renovate}
\subsection{dbdoc}
\subsection{OpenAPI}

\section{バージョン管理}

\subsection{asdf}
\subsection{node}
\subsection{poetry}

\section{その他}
\subsection{webhooks}
\subsection{deploykeys}

\section{おわりに}