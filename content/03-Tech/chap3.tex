\chapter{Gin × Neo4j × Docker で最短経路を返してくれるAPIサーバを建てる}

\section{Gin × Neo4j × Docker で最短経路を返してくれるAPIサーバを建てる}
\subsection{はじめに}
このセクションでは、現在最もメジャーなグラフDBであるNeo4jとGithubでも多くのスターを獲得しているGo言語のWebフレームワーク、Ginを用いて
APIサーバーを構築していきます。また、実行環境統一のためDockerを用います。

\section{今回扱うデータの図}
{../image/03-Tech/chap3/sample_node.png}

\section{ディレクトリ構造}
\begin{tcolorbox}[breakable]
    \begin{verbatim}
├── build
│   ├── Docker
│   │   ├── go
│   │   │   └── Dockerfile
│   │   └── neo4j
│   │       ├── Dockerfile
│   │       └── volumes
│   │           ├── import
│   │           │   ├── done
│   │           │   ├── points.csv
│   │           │   └── route.csv
│   │           └── script
│   │               └── import_data.sh
│   └── docker-compose.yml
└── server
    ├── config
    │   ├── config.go
    │   └── environments
    │       └── neo4j.yml
    ├── controllers
    │   └── coordinate_controller.go
    ├── db
    │   └── neo4j.go
    ├── go.mod
    ├── go.sum
    ├── main.go
    ├── models
    │   └── coordinate.go
    ├── router
    │   └── router.go
    └── sample.http
\end{verbatim}
\end{tcolorbox}

\subsection{Neo4jに最初にインポートするファイル}
point.csv
\begin{tcolorbox}[breakable]
    \begin{verbatim}
1 point_id:ID,point_name,:LABEL
2 a,PointA,Point;Position
3 b,PointB,Point;Position
4 c,PointC,Point;Position
5 d,PointD,Point;Position
6 e,PointE,Point;Position
\end{verbatim}
\end{tcolorbox}

route.csv
\begin{tcolorbox}[breakable]
    \begin{verbatim}
:START_ID,:END_ID,:TYPE,cost:int
b,a,Distance,1
c,b,Distance,3
d,c,Distance,2
e,d,Distance,2
c,a,Distance,1
d,a,Distance,2
e,b,Distance,1
e,a,Distance,1
c,e,Distance,1
d,b,Distance,3
a,b,Distance,1
b,c,Distance,3
c,d,Distance,2
d,e,Distance,2
a,c,Distance,1
a,d,Distance,2
b,e,Distance,1
a,e,Distance,1
e,c,Distance,1
b,d,Distance,3
\end{verbatim}
\end{tcolorbox}
これらをNeo4jにインポートすることで、先ほどの図を表現することができます。
{../image/03-Tech/chap3/neo4j_result.png}
無事できてますね。

\subsection{Docker環境の設定}
\begin{tcolorbox}[breakable]
    \begin{verbatim}
1 version: "3"
2 services:
3   go:
4     container_name: NEO4JAPI_GO
5     build:
6       context: ./docker/go
7       dockerfile: Dockerfile
8     stdin_open: true
9     tty: true
10    volumes:
11      - ../server:/server
12    ports:
13      - 8080:8080
14    networks:
15      app_net:
16        ipv4_address: 192.168.0.1
17    depends_on:
18      - "neo4j"
19
20  neo4j:
21    container_name: NEO4JAPI_NEO4J
22    build:
23      context: ./Docker/neo4j
24      dockerfile: Dockerfile
25    restart: always
26    ports:
27      - 57474:7474
28      - 57687:7687
29    volumes:
30      - ./Docker/neo4j/volumes/data:/data
31      - ./Docker/neo4j/volumes/logs:/logs
32      - ./Docker/neo4j/volumes/conf:/conf
33      - ./Docker/neo4j/volumes/import:/import
34      - ./Docker/neo4j/volumes/script:/script
35    environment:
36      - NEO4J_AUTH=neo4j/admin
37      - EXTENSION_SCRIPT=/script/import_data.sh
38    networks:
39      app_net:
40        ipv4_address: 192.168.0.2
41
42 networks:
43  app_net:
44    driver: bridge
45    ipam:
46      driver: default
47      config:
48        - subnet: 192.168.0.0/24
\end{verbatim}
\end{tcolorbox}
volumesはデータベースに入れるデータ、ログをバインドするための場所を指定しています。
environmentではユーザーとパスワード、最初に読み込んでもらうシェルスクリプトの場所を指定します。

go関連のDockerfile
\begin{tcolorbox}[breakable]
\begin{verbatim}
1  # goバージョン
2  FROM golang:1.19.3-alpine
3  # アップデートとgitのインストール
4  RUN apk add --update &&  apk add git
5  # appディレクトリの作成
6  RUN mkdir /server
7  # ワーキングディレクトリの設定
8  WORKDIR /server
9  # ホストのファイルをコンテナの作業ディレクトリに移行
10 ADD . /server
11 # main.goを実行
12 CMD ["go", "run", "main.go"]
\end{verbatim}
\end{tcolorbox}

Neo4j関連のDockerfile
\begin{tcolorbox}[breakable]
\begin{verbatim}
1  # goバージョン
2  FROM golang:1.19.3-alpine
3  # アップデートとgitのインストール
4  RUN apk add --update &&  apk add git
5  # appディレクトリの作成
6  RUN mkdir /server
7  # ワーキングディレクトリの設定
8  WORKDIR /server
9  # ホストのファイルをコンテナの作業ディレクトリに移行
10 ADD . /server
11 # main.goを実行
12 CMD ["go", "run", "main.go"]
\end{verbatim}
\end{tcolorbox}

csvファイルを読ませるためのシェルスクリプト
\begin{tcolorbox}[breakable]
\begin{verbatim}
1  #!/bin/bash
2  set -euC
3
4  # EXTENSION_SCRIPTはコンテナが起動するたびにコールされるため、
5  # import処理が実施済かフラグファイルの有無をチェック
6  if [ -f /import/done ]; then
7      echo "Skip import process"
8      return
9  fi
10
11 # データを全削除
12 echo "delete database started."
13 rm -rf /data/databases
14 rm -rf /data/transactions
15 echo "delete database finished."
16
17 # CSVデータのインポート
18 echo "Start the data import process"
19 neo4j-admin import \
20   --nodes=/import/points.csv \
21   --relationships=/import/route.csv
22 echo "Complete the data import process"
23
24 # import処理の完了フラグファイルの作成
25 echo "Start creating flag file"
26 touch /import/done
27 echo "Complete creating flag file"
\end{verbatim}
\end{tcolorbox}
    
\section{Goのソースコード}
\subsection{configディレクトリ}
config.go
\begin{tcolorbox}[breakable]
\begin{verbatim}
1  package config
2
3  import (
4  	  "github.com/spf13/viper"
5  )
6
7  var n *viper.Viper
8
9  func init() {
10 	 n = viper.New()
11	 n.SetConfigType("yaml")
12 	 n.SetConfigName("neo4j")
13 	 n.AddConfigPath("config/environments/")
14 }
15
16 func GetNeo4jConfig() *viper.Viper {
17	 if err := n.ReadInConfig(); err != nil {
18		 return nil
19	 }
20	 return n
21 }
\end{verbatim}
\end{tcolorbox}
このファイルでneo4j.yamlの環境変数を読み込みます。
\begin{tcolorbox}[breakable]
\begin{verbatim}
1 neo4j:
2  user: neo4j
3  password: admin
4  uri: neo4j://192.168.176.1:57687
\end{verbatim}
\end{tcolorbox}
Neo4jと接続するための環境変数です。
\subsection{dbディレクトリ}
neo4j.go
\begin{tcolorbox}[breakable]
\begin{verbatim}
1  package db
2
3  import (
4	  "log"
5
6	  "neo4japi/server/config"
7
8	  "github.com/neo4j/neo4j-go-driver/v4/neo4j"
9  )
10
11 func GetDriverAndSession() neo4j.Session {
12	 n := config.GetNeo4jConfig()
13	 dr, err := neo4j.NewDriver(n.GetString("neo4j.uri"), neo4j.BasicAuth(n.GetString("neo4j.user"), n.GetString("neo4j.password"), ""))
14	 if err != nil {
15		 log.Fatal(err)
16	 }
17	 ses := dr.NewSession(neo4j.SessionConfig{AccessMode: neo4j.AccessModeRead})
18	 return ses
19 }
20
\end{verbatim}
\end{tcolorbox}
このファイルでNeo4jと接続し、セッションを返すようにします。
\subsection{modelsディレクトリ}
coordinate.go
\begin{tcolorbox}[breakable]
\begin{verbatim}
1  package models
2
3  import (
4	  "fmt"
5	  "log"
6
7	  "neo4japi/server/db"
8
9	  "github.com/neo4j/neo4j-go-driver/v4/neo4j"
10 )
11
12 type Route struct {
13	 Position string `json:"point"`
14 }
15
16 func FindRoute(fr, to string) []*Route {
17 	 var r []*Route
18	 ses := db.GetDriverAndSession()
19	 defer ses.Close()
20	 cyp := fmt.Sprintf(`
21		 MATCH (from:Position {point_name: "%s"}), (to:Position {point_name: "%s"}), 
22			 path=allShortestPaths ((from)-[distance:Distance*]->(to))
23		 WITH
24			 [position in nodes(path) | position.point_name] as name,
25		 REDUCE(totalMinutes = 0, d in distance | totalMinutes + d.cost) as 所要時間
26		 RETURN name
27		 ORDER BY 所要時間
28		 LIMIT 10;
29	 `, fr, to)
30
31	 _, err := ses.ReadTransaction(func(transaction neo4j.Transaction) (interface{}, error) {
32		 result, err := transaction.Run(cyp, nil)
33		 if err != nil {
34			 return nil, err
35		 }
36		 if result.Next() {
37			 name, _ := result.Record().Get("name")
38			 nameAr := name.([]interface{})
39
40			 for i := 0; i < len(nameAr); i++ {
41				 r = append(r, &Route{nameAr[i].(string)})
42			 }
43		 }
44		 return nil, result.Err()
45	 })
46	 if err != nil {
47		 log.Fatal(err)
48	 }
49	 return r
50 }
\end{verbatim}
\end{tcolorbox}
出発地点と目的地を受け取ることでNeo4jにCypherというクエリ言語を用いて最短経路を導出してもらいます。
\subsection{controllersディレクトリ}
coordinate_controller.go
\begin{tcolorbox}[breakable]
\begin{verbatim}
1  package controllers
2
3  import (
4	  "net/http"
5
6	  "github.com/gin-gonic/gin"
7	  "neo4japi/server/models"
8  )
9
10 func RouteSearch(c *gin.Context) {
11	 fr := c.Query("fr")
12	 to := c.Query("to")
13	 res := models.FindRoute(fr, to)
14	 c.JSON(http.StatusOK, res)
15 }
\end{verbatim}
\end{tcolorbox}
GETリクエストで受け取ったパラメータを models で作成した関数に渡します。
\subsection{routerディレクトリ}
router.go
\begin{tcolorbox}[breakable]
\begin{verbatim}
1  package router
2
3  import (
4	  "github.com/gin-gonic/gin"
5	  "neo4japi/server/controllers"
6  )
7
8  func Init() {
9	  r := gin.Default()
10	 r.GET("/coordinate", controllers.RouteSearch)
11	 r.Run()
12 }
\end{verbatim}
\end{tcolorbox}
ルーティング先を定義します。
\subsection{実行ファイル}
coordinate_controller.go
\begin{tcolorbox}[breakable]
\begin{verbatim}
1  package main
2
3  import (
4	  "neo4japi/server/router"
5  )
6
7  func main() {
8	  router.Init()
9  }
\end{verbatim}
\end{tcolorbox}

\section{実行方法}
\begin{tcolorbox}[breakable]
\begin{verbatim}
docker compose up
\end{verbatim}
\end{tcolorbox}
このコマンドを実行することで Neo4j サーバーと Gin サーバーが立ち上がります。
\subsection{実行確認}
sample.http
\begin{tcolorbox}[breakable]
\begin{verbatim}
1  GET http://localhost:8080/coordinate?fr=PointB&to=PointE
\end{verbatim}
\end{tcolorbox}
ここで、 coordinate?fr=PointB&to=PointE の部分を自分の好きな地点にしてリクエストを送ると、最短経路が返されます。