\chapter{リポジトリ作成後に設定しておきたいこと}
\section{はじめに}
こんにちは。hihumikanです。

本チャプターでは「リポジトリ作成後に設定しておきたいこと」をご紹介します。
私自身が次回、プロジェクトを開始する際に、こういった設定や技術を使うだろうというものをまとめました。

ただし、これら全ての内容をプロジェクトに適用出来るものではないため、ご利用の環境や用途に合わせて利用していただければと思います。

\subsection{対象読者}

対象読者は、Git/GitHubを利用した開発を行う初学者の方を想定しています。

\section{ファイルの設定}

リポジトリ作成後に設定しておきたい「ファイルの設定」についてご紹介します。

\subsection{.gitignore}

.gitignoreは、Gitによる管理から除外したいファイルやディレクトリを指定するための設定ファイルです。

例えば、MacOSの場合、ディレクトリ毎に.DS\_Storeというファイルが自動的に生成されます。
このファイルは、ディレクトリのmeta情報を記録しますが、通常の開発において共有する必要がないため、Gitの管理対象外としておくのが望ましいです。

また、.envなどの環境変数を利用してプログラムを動かす場合、.envファイルには、パスワードや秘密鍵などの情報が含まれているのが多いため、これもGitの管理対象外としておくのが望ましいです。

リスト1.1のようなテキストファイルをGitの管理下に置くだけで、Gitから管理対象外として扱われます。

\begin{tcolorbox}[title=リスト1.1 .gitignore]
  \begin{verbatim}
1 .DS_Store
2 .env
\end{verbatim}
\end{tcolorbox}

リポジトリを共有する場合、.gitignoreファイルを置いておくだけで、開発メンバー全員が同じ設定を利用できるため、必要なファイルだけがGitの管理下に置かれるようになります。

\subsection{.gitattributes}

.gitattributesは、特定のファイルに対してGitの挙動を変更するための設定ファイルです。
主に、改行コードやファイルの文字コードを指定するために利用されます。

改行コードを指定する理由としては、WindowsとMacOSでは改行コードが異なる点が挙げられます。
Gitで管理されているファイルの改行コードが一致しないと、差分が発生してしまい、想定した挙動と異なる動作をする可能性があります。
それを防ぐために、.gitattributesに改行コードを明示しておくことで、安全に開発を進めれます。

リスト1.2のように、ファイルの拡張子に対して、改行コードを指定が出来ます。

\begin{tcolorbox}[title=リスト1.2 .gitattributes]
  \begin{verbatim}
1 * text=auto
2 *.sh text eol=lf
\end{verbatim}
\end{tcolorbox}

これも.gitignoreと同様に.gitattributes共有するだけで、開発メンバー全員が同じ設定を利用できます。


\subsection{Makefile}

Makefileは、コマンドをまとめて実行するための設定ファイルです。

利点として、開発メンバー全員が同じコマンドを実行出来る所にあります。
環境構築やテストの実行など、手順が複雑な作業を一人一人が実行した場合、手順の違いによるエラーが発生する可能性があります。
それらを防ぐために、Makefileにまとめておくことで、開発メンバー全員が同じコマンドを実行でき、人的ミスを防げます。

Makefileの例としては、リスト1.3のようなものです。

\begin{tcolorbox}[title=リスト1.3 Makefile]
  \begin{verbatim}
1 up: ## APIとデータベースを起動
2   docker compose -f docker-compose-db.yml -p db up -d
3   docker compose -f docker-compose-api.yml -p api up -d
4 
5 build: ## サービスの構築
6   docker compose -f docker-compose-db.yml -p db build
7   docker compose -f docker-compose-api.yml -p api build
8 
9 stop: ## サービスを停止
10   docker compose -f docker-compose-db.yml -p db stop
11   docker compose -f docker-compose-api.yml -p api stop
12 
13 kill: ## サービスを強制停止
14   docker compose -f docker-compose-db.yml -p db kill
15   docker compose -f docker-compose-api.yml -p api kill
16 
17 down: ## サービスの停止とコンテナの削除
18   docker compose -f docker-compose-db.yml -p db down
19   docker compose -f docker-compose-api.yml -p api down
20 
21 restart: ## サービスの再起動
22   docker compose -f docker-compose-db.yml -p db restart
23   docker compose -f docker-compose-api.yml -p api restart
 \end{verbatim}
\end{tcolorbox}

これらを実行する場合、シェルに
\begin{shaded}
  \begin{verbatim}
$ make up
\end{verbatim}
\end{shaded}
と入力するだけで、長いコマンドであった2,3行目のコマンドが簡単に実行されます。
リスト1.3の例は簡単なものですが、他にもMakefile内に変数を定義できるため、変更が必要な箇所を変数に置き換えることで、コマンドの変更を容易に行えます。

\section{インフラ周り}

\subsection{GitHub Actions}
Github Actionsは、GitHub上で動作するCI/CDツールです。
簡単に言えば、GitHubリポジトリに関連するイベントに応じて、あらかじめ定義しておいたワークフローを仮想マシン上で実行するものです

用途としては、コードの静的解析やテストの実行、デプロイなどが挙げられます。
その他にも、Pull Requestに対して、テスト内容をコメントしてくれるなどのGitHub上での機能を利用できます。

利用方法としては、リスト1.4のように、.github/workflowsディレクトリを作成し、その中に設定ファイルを作成します。

\begin{tcolorbox}[title=リスト1.4 .github/workflows/pr.yml]
  \begin{verbatim}
1 on:
2   pull_request:
3     types: [opened]
4 name: Pull Request
5 jobs:
6   assignAuthor:
7     name: Assign author to PR
8     runs-on: ubuntu-latest
9     steps:
10      - name: Assign author to PR
11        uses: technote-space/assign-author@v1
\end{verbatim}
\end{tcolorbox}

上記の例では、Pull Requestが作成された際に、Pull Requestの作成者をAssigneeに設定する設定ファイルの例です。

その他にも、リスト1.5のように、ssh鍵を設定することで、リモートサーバーにアクセスができます。
\begin{tcolorbox}[title=リスト1.5 .github/workflows/deploy.yml]
  \begin{verbatim}
1 name:CI
2 on:
3   push:
4     branches:
5       - main
6 jobs:
7   deploy:
8     runs-on: ubuntu-latest
9     steps:
10      - uses: actions/checkout@v2
11      - name: Install SSH Key for Deploy
12        uses: appleboy/ssh-action@master
13        with:
14          key: ${{ secrets.SK }}
15          host: ${{secrets.SSH_HOST}}
16          username: ${{secrets.SSH_USERNAME}}
17          port: ${{secrets.SSH_PORT}}
18          script: |
21            git pull
\end{verbatim}
\end{tcolorbox}

この他にも、コードを自動で整形してcommitしてくれるツールなどの様々なツールが存在します。
調べてみると面白いかもしれません。

\section{おわりに}
本チャプターでは「リポジトリ作成後に設定しておきたいこと」をご紹介しました。
ご紹介したのは一部分ですが、これらを設定しておくことで、開発の効率化や、開発メンバーのミスを防げます。
ぜひ、設定してみてください。
