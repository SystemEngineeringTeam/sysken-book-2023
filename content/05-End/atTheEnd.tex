\newpage
\thispagestyle{empty}
\section*{奥付け}

今回はこの本を手に取っていただきありがとうございました。
企画・編集を行いました、牧野です。
さて、10年ほど前のシス研では、本という形で技術をアウトプットするという文化があったと先輩から教えてもらいました。
今では、QiitaやZennのような技術ブログが発達して、物理的なもので残す文化がちょっとずつ廃れてしまったらしいです。
そんな中、私がC101の冬コミで技術本というものに出会いました、一つ一つ個人で調べたり研究した情報を、本という形で残すことにとてもいいと思い、今回このような本を執筆しました。
本というものは知識としてずっと蓄えられるものですので、私たちが書いた本が誰かの役にたてればと思い、締めさせてもらいます。
次回も余裕があれば出したいな。

企画・編集 牧野遥斗

\begin{table}[b]%
	\centering%
	\begin{tabular}{lcll}%
		\multicolumn{4}{c}{ {\LARGE Syskenの技術本 様々な技術を詰め合わせてみました。} }	\\
		\bhline{1pt}
		発行日 && 2023年 5月 28日 & (初版)	\\
%		 && 2019年 $\phantom{1}$2月 28日 &	(第二版)\\
		サークル && 愛知工業大学 システム工学研究会 &	\\
    Instagram ID && @ait.sysken& 	\\
		Twitter ID && @set$\_$official &	\\
		QiitaOrganizationURL && https://qiita.com/organizations/sysken &	\\
		代表 && 牧野遥斗 & \\
		代表者メールアドレス && harutiro2027@icloud.com & \\
		企画・編集 && 牧野遥斗 (Twitter: @minesu1224)  &	\\
		著者 && 林航平  &	\\
		   && suda michiyo  &	\\
		   && hihumikan  &	\\
		   && Beyond Toyama  &	\\
		   && 水谷祐生 (Twitter @l8ZAFNZbON1eDbZ)  &	\\
		   && BlacKnight松土 (Twitter:@kk22blacknight)  &	\\
		   && shirataki1126  &	\\
		印刷所 && しまや出版 & \\
		\bhline{1pt}
		\multicolumn{4}{c}{ {※本書の無断複写、複製、データ配信はかたくお断りいたします。} }	
	\end{tabular}%
\end{table}%