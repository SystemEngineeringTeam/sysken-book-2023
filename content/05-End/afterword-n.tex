% TODO: chapterやsectionは適宜修正お願い
\chapter{あとがき枠}

\section{本誌創刊に寄せて}

物は試し、ということわざがあります。
また、言うは易く行なうは難し、ということわざもあります。
どちらのことわざにしても、実際にやってみれば良く分かる、という共通の帰結があります。
しかし、あまりに簡単に実行できる仕組みを作ってしまうと、しばしば人間は仕組み無しで試すことや仕組み自体を軽視し始め、その仕組みの存在のありがたさを忘れてしまうものなのでしょう。
そして、失って初めてその存在のありがたみに気が付く、という言葉は、前述のような体験を経た (やってみた) 人からしばしば生まれてくる言葉なのでしょう。
では、あまりに簡単に成功する仕組みがある状況下で、次代にその仕組みの意味や価値を伝えるにはどうすればよいのでしょうか。

次代の人々に対して、比較的短期間かつ効果が期待できる方法のひとつに、意図的な損失を生み出す方法があります。
失って気が付くのであれば一度失ってみさせれば良い、という考えです。
しかし、一度得たものを失うことに少なからず不満を抱いてしまうのは、人間の性です。
故に、実行者にはその不満の矛先が向けられる可能性があります。
また、損失は発展を阻害する要因でもあります。
従って、進歩のための損失であるように、損失の度合いには十分に配慮しなければなりません。

一方で、正義は常に正しいとは限らない、という考えがあります。
それは、時代や地域や環境によって、文化や思想や理念など、是非を判断する上での前提条件が異なるからです。
ある正義の下では正当だとされる仕組みでも、別の正義の下では不当だとされ得るということです。
しかしながら、正義は人間の行為における動機付けのひとつとされています。
昨日までの正義を否定するような今日の正義に直面した時、我々はどのように受け止め考えていけば良いのでしょうか。
選択肢として、一切の拒絶をするか、昨日までの自らの行いを否定することによって自己正当化をするか、などが考えられます。
それらの中でも、多少なりとも理解を試み受け入れようとする選択が、多角的な視点からより良い考えが期待できるでしょう。

私はシス研に長らく在籍していました。
本誌のような試みが10年振りに復活し、製本されることを大変嬉しくありがたく思っています。
シス研では、ある時は観察者として、またある時は友や助言者として、そして敵として、振舞いました。
かつて理想と信念を掲げ、仲間と共にある時代を築いた者のひとりとして、自主的に行ったとはいえ、次代の芽が出るまでの橋渡し役はなかなかに堪えるものでした。
自らが信じた正義と次代を担う彼らの正義を常に見比べ、その上で役割を果たす必要があったからです。
何かを手に入れようとする若さを生かすためには、何かを失うまいとする老いた私情は捨て去らねばなりません。
しかしながら、過去の歴史を知り同じ過ちを繰り返さないようにすることは、未来を思い描くために重要なことであるはずです。
ある意味ではそれを正義として、客観的な答えを導くために私は自分自身を自制していたのかもしれません。
少なくとも、未来を思い描く先導者にとって、彼ら自身が追従するような相手はいませんから。

これから、シス研は新たな時代を迎えようとしています。
最後に、マハトマ・ガンジーが残した2つの名言で締めつつ、シス研の今後の活動にご期待ください。

\begin{quote}
    物事は初めはきまって少数の人によって、ときにはただ一人で始められるものである。
    \\
    満足は努力の中にあって、結果にあるものではない。
\end{quote}
