\chapter{シス研というサークルについて}

\section{シス研というサークルについて}
\subsection{はじめに}
初めまして!シス研会長の林です!
はい、ここでみなさんシス研とはなんぞ?となっていると思うのでまずは自分が水先案内人となりましてこの本とシス研について解説していこうと思います。

\subsection{どんな話をするのか}
シス研ってどんなサークル?どんな活動をしているの?この本はどう言ったもの?といったものを紹介していきます。それでは、さっそく行ってみましょう!!!

\section{シス研とは}
シス研は正式名称を「システム工学研究会」と言い、愛知工業大学公認の情報系サークルです。歴史は長く、1976年に成立しており、かのAppleと同い年となります!!
シス研ではハッカソン出場をはじめとしたチーム開発、ゲーム作成、インフラ整備を主な活動内容としています。

\subsection{どんな活動をしているの?}
シス研の主な活動はチーム開発とインフラ整備です。サークル全体としての開発物などはなく、それぞれがチームを組んでハッカソンに出場したりしています。
インフラ面では、部室に物理サーバを持っており、そこでシス研のホームページや各種サービスを公開しています。\footnote{シス研ホームページ\url{https://set1.ie.aitech.ac.jp}}\footnote{シス研紹介ページ\url{https://welcome.sysken.net}}23年4月現在、大幅な工事を行なっておりごく一部のサービスのみ稼働しています(すみません)
そのほかにもシス研主催のLT会、ハッカソン、Qiitaアドベントカレンダーへの参加もしています。\footnote{Advent Calendar 2022\url{https://qiita.com/advent-calendar/2022/stech-ait-advent}}

\subsection{活動実績}
\begin{itemize}
  \item 2021 愛工大大学祭 工科展 最優秀賞
  \item 2022 Geekcamp vol8 優秀1
  \item 2022 愛工大大学祭 工科展 瑞若翔
  \item 2022 愛工大大学祭 模擬店 最優秀賞
\end{itemize}

\begin{tcolorbox}[title=シス研の設備]
  コラムも随時挟めるようにした。

  tcolorboxはtitleを指定するといい感じにタイトル付きの枠で囲ってくれる。
\end{tcolorbox}